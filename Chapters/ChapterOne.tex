\chapter{Introdução}
\label{chap:intro}
\begin{flushright}
	"Faça ou não faça, tentativa não há." \\
	\ \\
	(Mestre Yoda)
\end{flushright}

Quando se ouve falar em robôs, logo associa-se a algo de extrema complexidade. Isso ocorre, sumariamente, devido à falta de informações simplificadas sobre o tema, ou devido a dificuldade de acesso a tais conteúdos. A palavra robótica é derivada da palavra robô, que, segundo \cite{goncalves2007}, é um dispositivo eletromecânico capaz de realizar tarefas de maneira autônoma ou pré-programada, e faz menção a ciência que estuda, cria e aplica robôs. 
No meio educacional, a palavra didática está presente de forma quase que impreterível, afinal, materiais didáticos, livros, projetos e a própria didática como um instrumento qualificador do professor, são componentes fundamentais do cotidiano educacional. Porém é notório que barreiras na educação da atualidade estão sendo quebradas, onde o estudante deixa de frequentar as salas de aula, tornando assim o professor apenas um facilitador do aprendizado do aluno, um tutor. A tutoria é um método muito utilizado para efetivar uma interação pedagógica. Segundo \cite{sa1998}, na educação à distância, o tutor recebe o significado de "orientador de aprendizagem do aluno solitário e isolado".
O sistema de tutoria torna mais fácil o acesso do aluno ao conhecimento, pois o professor passa a ser apenas um orientador, desta maneira aluno torna-se independente na busca das informações, assim, simplificando o aprendizado do aluno. Percebendo essa nova dinâmica da educação, e a falta de informações simplificadas sobre robótica, notou-se a possibilidade de criar um kit didático, para incentivar as pessoas através de desafios e simplificar as informações em torno da robótica.




%--------- NEW SECTION ----------------------
\section{Organização do \thetypework}
\label{section:organizacao}
O documento está organizado em cinco capítulos, seguindo a seguinte estrutura:

\textbf{Capitulo 1 - Introdução}: Faz a contextualização do âmbito no qual a pesquisa proposta
está inserida. Apresenta, portanto, a problemática, objetivos e como este projeto Theoprax de conclusão de curso está estruturado


\textbf{Capítulo 2 - Referencial Teórico}: Apresenta a base teórica necessária para o desenvolvimento do projeto.

\textbf{Capítulo 3 - Metodologia}: Define o método adotado para o desenvolvimento do projeto, explicitando seu fluxo de atividades e premissas necessárias para aplicar a metodologia.

\textbf{Capítulo 4 - Desenvolvimento}: Exibe os procedimentos realizados e resultados obtidos através de testes, unitários e integrados, durante o desenvolvimento do projeto.

\textbf{Capítulo 5 - Conclusão}: Apresenta as conclusões, contribuições e algumas sugestões de atividades de pesquisa a serem desenvolvidas futuramente.


